\documentclass[a4paper,12pt,bold,leqno,fleqn,a4]{scrartcl}

\renewcommand{\baselinestretch}{1.5}\normalsize

\usepackage{apacite}
\usepackage[round,longnamesfirst]{natbib}
\usepackage{bm}																									%matrix symbol
\usepackage{setspace}																						%Fu�noten (allgm.
\usepackage{hyperref}
%Zeilenabst�nde)
\usepackage{threeparttable}
\usepackage{lscape}																							%Querformat
\usepackage{graphicx}
\graphicspath{{material/}}

\usepackage{amsmath}
\usepackage{amssymb}
\usepackage{fancybox}																						%Boxen und Rahmen
\usepackage{appendix}
\usepackage{enumerate}
																					%EURO Symbol
\usepackage{tabularx}
\usepackage{longtable}																					%Mehrseitige Tabellen
\usepackage{color,colortbl}																			%Farbige Tabellen
\usepackage[left=2.5cm, right=2.5cm, top=2cm, bottom=3cm]{geometry} %Seitenr�nder
%\usepackage[normal]{caption2}[2002/08/03]												%Titel ohne float - Umgebung

\setlength{\skip\footins}{1.0cm}
\deffootnote[1em]{1.1em}{0em}{\textsuperscript{\thefootnotemark}}
\renewcommand{\arraystretch}{1.05}



\newenvironment{boenumerate}
{\begin{enumerate}\renewcommand\labelenumi{\textbf{(\theenumi)}}}
{\end{enumerate}}
\parindent0pt


\title{Potential Outcome Model}
\subtitle{-- Problem Set 1 --}
\date{}

\begin{document}\maketitle\vspace{-2cm}

The \href{https://www.cdc.gov/nchs/nhis/index.htm}{\texttt{National Health Interview Survey (NHIS)}} data is collected on U.S. households since 1957; it covers a broad range of health-related topics from medical conditions, health insurance, and the number of doctor visits to measures of physical activity. Here we focus on indicators relevant for the POM framework; in particular, we will compare the health status of hospitalized and non-hospitalized individuals in 2018.
For this purpose, we use answers to the survey question ``During the past 12 months, has the respondent been hospitalized overnight?`` with potential answers ``yes'' and ``no`` which we code as 1 and 0. Further, we consider answers to the questions ``Would you say your health, in general, is excellent, very good, good, fair, poor?`` where responses are coded as 1 for poor health up to 5 for excellent health. The survey also collects data on relevant characteristics as sex, age, level of education, hours worked last week, and total earnings.

\section*{Task A} 

\begin{boenumerate}

  \item Open a \texttt{Jupyter Notebook} and import the data set \href{https://github.com/HumanCapitalAnalysis/microeconometrics/tree/master/problem-sets/01-potential-outcome-model/data}{\texttt{nhis-initial.xslx}} --- we recommend using the software presented in class (e.g. \texttt{pandas}, \texttt{matplotlib}, etc.). Try to think of ways to answer the following questions: \emph{Are there more females or males?} \emph{Are there more individuals who hold a degree or not?}. Now try to relate individual characteristics to the hospitalization status. \emph{Are high or low earners/old or young people more often hospitalized?}

  \item Compute the average health status of hospitalized and non-hospitalized individuals. \emph{Who is healthier on average?} \emph{What could be a reason for this bias?}

  \item Adjust the data set for the POM framework (as seen in the lecture), with health status as the outcome and hospitalization as the treatment status (Hint: rename and drop columns of the data frame).

  \item Compute the naive estimate for the \emph{average treatment effect} (ATE).

\end{boenumerate}


\section*{Task B}

\begin{boenumerate}
  \item As we've seen in the lecture, in reality, we can only ever observe one counterfactual; however, when simulating data, we can bypass this problem. The (simulated) data set  \href{https://github.com/HumanCapitalAnalysis/microeconometrics/tree/master/problem-sets/01-potential-outcome-model/data}{\texttt{nhis-simulated.xslx}} contains counterfactual outcomes, i.e., outcomes under control for individuals assigned to the treatment group and vice versa. Derive and compute the average outcomes in the two observable and two unobservables states; Design them in a similar way as Table 2.3 in \cite{Morgan.2014}.

\end{boenumerate}

\noindent From here on we assume that 5$\%$ of the population take the treatment.

\begin{boenumerate}\setcounter{enumi}{1}

\item Derive and explain formula 2.12 from \cite{Morgan.2014} for the naive estimator as a decomposition of true ATE, baseline bias, and differential treatment effect bias (do this with pen and paper).

\item Compute the naive estimate and true value of the ATE for the simulated data. \emph{Is the naive estimator upwardly or downwardly biased?} Calculate the baseline bias and differential treatment effect bias. \emph{How could we interpret these biases in our framework of health status of hospitalized and  non-hospitalized respondents?}

\item Which assumptions must hold (on the data) such that the naive estimator is an \emph{unbiased} and \emph{consistent} estimator for the ATE?

\end{boenumerate}

\nocite{Angrist.2008}
\nocite{NHIS}

\bibliographystyle{apacite}
\bibliography{../../../submodules/bibliography/literature}

\end{document}
